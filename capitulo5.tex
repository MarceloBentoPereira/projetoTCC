%========================= CAPITULO 5 ==============================

\chapter{Conclusão e Trabalhos Futuros} %\label{cap_exemplos}
%\thispagestyle{empty}
Através deste projeto foi possível, a partir dos estudos teóricos e práticos sobre a telefonia convencional, VoIP, e a ferramenta Asterisk, elaborar, configurar e testar através de um protótipo, ou seja, demonstrar que é possível aplicar inteligencia computacional neste teleatendimento e ainda proporcionar segurança e confiabilidade aos atendentes e cidadãos douradenses com relação ao teleatendimento da Polícia Militar.

O trabalho desempenhado ainda demonstrou, além de um estudo teórico para levantamento bibliográfico, o desenvolvimento prático de uma servidor PABX, é capaz de prover comunicação em um ambiente misto de telefonia, onde o usuário, localizado em algum ponto da telefonia pública (PSTN), se comunica com a rede de dados com ramais VoIP. A partir do efetivo funcionamento deste servidor foi possível, de maneira prática, agregar valor ao sistema de telefonia do teleatendimento.

Este trabalho conseguiu contribuir no sentido de enriquecer e agregar conhecimento a este tema, até então, pouco explorado pela comunidade científica. Demonstrou-se também inovador, pois foi elaborado um protótipo de servidor com funcionalidades pertinentes a um setor de teleatendimento emergencial de um órgão de segurança publica.

A implementação do protótipo no teleatendimento foi autorizado pelo diretor técnico do CIOPS, o qual acompanhou alguns testes realizados com o protótipo. E após esta etapa, o cenário apresentado foi aprovado pelo mesmo e os integrantes do teleatendimento que puderam acompanhar os testes realizados.

Como trabalhos futuros, sugere-se que sejam realizados testes no ambiente, fazendo uso de hardware específico para a comunicação do ambiente baseado em VoIP com a telefonia convencional, afim de verificar a sua estabilidade e eficiência. Bem como a ampliação do número de informações monitoradas no servidor.

Sugere-se também que sejam realizados testes de integração com o servidor VoIP e as viaturas que possuírem algum dispositivo móvel conectado a internet, para verificar a possibilidade de teleconferência entre os atendentes e as viaturas bem como o cidadão a qual originou a chamada ao teleatendimento.

Com o ambiente de telefonia em pleno funcionamento, sugere-se também o desenvolvimento de um número maior de aplicações, como a personalização da bilhetagem, onde mais informações podem ser guardadas. E a possibilidade da criação ou mesmo reaproveitamento de softphone com código fonte aberto, para personalização para teleatendimento, a exemplo onde após um atendimento o atendente possa colocar a chamada como sendo realmente um atendimento, trote ou pedido de informação.

