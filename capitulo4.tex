%========================= CAPITULO 4 ==============================

\chapter{Aplicação} %\label{cap_exemplos}
%\thispagestyle{empty}
\section{Instalação do Asterisk}
A instalação do Asterisk requer e exige uma série de cuidados e detalhes,  para que se tenha um ambiente de telefonia estável e disponível o maior tempo possível. Logo, e necessário uma instalação onde busca-se a menor quantidade de problemas e com maior tempo operando, pois a compilação do Asterisk não é um processo trivial \cite{alexandrekeller2014}.

Apesar de haver duas formas de instalação, onde pode ser por SVN\footnote{Sistema de controle de versão} que mantêm um compartilhamento do desenvolvimento do sistemas com todas as novas funcionalidades, porém ainda nem sempre devidamente testadas, logo optou-se por baixar os pacotes compactados sendo eles:

\begin{itemize}
  \item dahdi-linux-current.tar.gz \newline http://downloads.asterisk.org/pub/telephony/dahdi-linux/
  \item dahdi-tools-current.tar.gz
  \item libpri-1.4-current.tar.gz
  \item openr2-1.3.3.tar.gz, obtido através do endereço: \newline https://code.google.com/p/openr2/downloads/list
  \item libss7-1.0.2.tar.gz
  \item asterisk-13-current.tar.gz
\end{itemize}
