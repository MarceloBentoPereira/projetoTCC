%========================= CAPITULO 1 ==============================
%INTRODUÇÃO, OBJETIVOS, JUSTIFICATIVAS/MOTIVAÇÃO, METODOLOGIA, ORGANIZAÇÃO DO TEXTO

\chapter{Introdução}%Cap.1
%\thispagestyle{empty}
%\addcontentsline{toc}{chapter}{Introdução}
Em sociedade sempre necessitamos de serviços, sejam eles públicos ou privados, mas quando ocorre a necessidade de um serviço de um órgão da segurança pública, geralmente ele não está presente nas proximidades, e o meio mais preciso e ágil de acioná-lo é através do telefone emergencial que este disponibiliza.

Os órgãos de segurança pública como qualquer empresa ou instituição necessitam do sistema de telefonia para sua própria comunicação interna ou externa, bem como atender por meio de números emergenciais como, por exemplo, o número emergencial 190 da Polícia Militar.

Um cidadão ao efetuar uma ligação para um número emergencial, precisa ser atendido com eficiência. Desse modo, as linhas emergenciais devem estar disponíveis 24 horas por dia, isto é, não devem haver linhas defeituosas, congestionadas ou comprometidas, seja por falha técnica ou humana. No entanto, essas falhas existem.

No município de Dourados, o comandante Ary Carlos Barbosa do 3º Batalhão de Polícia Militar relata em um artigo jornalistico\footnote{http://www.douradosnews.com.br/dourados/populacao-reclama-do-nao-funcionamento-do-190-da-pm}:

\begin{citacao}
``Um ramal dos telefones da emergência está mudo, e os técnicos não conseguem identificar o problema, para a pessoa que está ligando, ela escuta chamando, mas para a gente não. Já chamamos à operadora, trocaram a fiação, mas não acham a linha que está com problema, às vezes ligam e cai em Deodápolis, Douradina''.
\end{citacao}

O problema relatado pelo comandante aparentemente é técnico, porém também pode se referir a uma falha humana, pois na mesma reportagem a autora \citeonline{eduardarosa2014} em seu artigo jornalístico descreve o depoimento de outras pessoas entrevistadas que disseram que tentaram por algumas vezes o atendimento e não conseguiram ser atendidas.

Uma possível melhoria tecnológica no teleatendimento da Polícia Militar é através do emprego de uma tecnologia que está em constante evolução e expansão, como por exemplo o VoIP. 

A VoIP permite o tráfego de voz sobre redes de computadores, sendo possívelefetuar e receber ligações entre computadores, telefones comuns, telefones IP e celulares. Pode também ser utilizada como centrais de PABX \textit{(Private Automatic Branch eXchange)}, que podem ser substituídas por servidores de PABX IP \textit{(Private Automatic Branch eXchange Internet Protocol)} \cite{glauciadasilvaribeiro2011}.

Embora a VoIP tenha melhorias por utilizar da rede de dados para trafegar voz, é inegável que a rede de telefonia convencional possui uma abrangência maior, com sua estrutura envolvendo todas as localidades do planeta. Essa é uma grande vantagem que a tecnologia empregada na rede de telefonia pública comutada (RTPC), em inglês PSTN \textit{(Public Service Telephony Network)} possui sobre a tecnologia de VoIP \cite{theodorewallingford2005}.

Portanto, o Asterisk\footnote{Software de tefonia de licença gratuita, disponibilizado pela empresa Digium.} é considerado uma central de telefonia híbrida, pois tem como benefício o emprego da tecnologia de telefonia convencional, como a tecnologia VoIP, logo, este trabalho tem como proposta, proporcionar uma melhoria no teleatendimento da Polícia Militar, utilizando a ferramenta Asterisk por meio de uma central de telefonia híbrida.

% ================= Objetivos =================================
\section{Objetivos}
\subsection{Objetivo Geral}
Propor uma melhoria do teleatendimento da Polícia Militar de Dourados através do desenvolvimento de um protótipo de um servidor VoIP, que implemente tanto as funções de telefonia convencional, quanto a tecnologia VoIP, com o emprego das funcionalidades que a ferramenta Asterisk proporciona.

\subsection{Objetivos Específicos}
Dado o domínio do teleatendimento emergencial da Polícia Militar de Dourados, os objetivos específicos são:

\begin{itemize}
	\item Explorar o problema através de aplicação de questionários aos atendentes;
	\item Ter o emprego de funcionalidades de PABX no teleatendimento.
	\item Utilizar de um SGDB para armazenar as informações de todas as chamadas recebidas pela central telefônica.
	\item Realizar levantamento estatístico dessas informações.
\end{itemize}

%================== Justifica\c{c}\~{a}o/Motiva\c{c}\~{a}o =====================
\section{Justificativa/Motivação}
Devido as constantes reclamações da população douradense sobre qualidade e funcionamento do teleatendimento da Polícia Militar destinados ao diretor técnico do CIOPS\footnote{Órgão onde concentra-se o teleatendimento emergencial da Polícia Militar de Dourados.} (Centro Integrado de Operações de Segurança Pública) e a possível deficiência técnica ou até mesmo humana no teleatendimento emergencial da Polícia Militar de Dourados. Tem-se como motivação para este trabalho uma proposta de melhoria nesse teleatendimento emergencial, explorando uma alternativa viável e acessível como a tecnologia VoIP que está em constante evolução e expansão, bem como, o emprego de uma ferramenta de licença gratuita como o Asterisk que agrega uma gama enorme de funcionalidades a um ambiente de telefonia convencional, trazendo não tão somente conforto e segurança ao atendentes, bem como, confiança aos cidadãos douradenses para este teleatendimento.

\section{Metodologia}
A metodologia envolveu as seguintes etapas:
\begin{enumerate}
  \item Fundamentação teórica com estudos sobre a telefonia convencional, seu funcionamento, organização e infraestrutura, bem como, seus tipos de sinalização e estabelecimento de circuitos.
  \item Descrever os conceitos gerais sobre VoIP, seu histórico, tipos de redes, protocolos utilizados tanto na sinalização como na transmissão e principais benefícios ao empregar esta tecnologia.
  \item Apresenta detalhes da ferramenta escolhida, o Asterisk, bem como, suas funcionalidades, os módulos necessários para sua utilização, questão de hardware e sua arquitetura.
  \item Avaliar as principais falhas técnicas apontadas pelos atendentes e observadas no período de permanência no teleatendimento emergencial da Polícia Militar, bem como, realizar a análise de requisitos necessários para a aplicação do protótipo do servidor VoIP.
  \item Relatar a aplicação, configurações e os testes do protótipo ao ser empregrado no teleatendimento, e apresentar as considerações sobre os resultados obtidos.
\end{enumerate}

Detalhando a metodologia, tem-se na primeira etapa, a fundamentação teórica. Onde exibira os conceitos sobre o sistema de telefonia pública comutada (PSTN), ou seja, o atual sistema de telefonia que conhecemos, sistema esse que serviu de base para criação da tecnologia de telefonia baseado em VoIP.

	Na segunda etapa, descrevera sucintamente o início, motivação, evolução da tecnologia VoIP, bem como, seus protocolos utilizados tanto na sinalização como a transmissão da voz em redes TPC/IP, consequentemente na internet, ainda como vem sendo empregada, utilizada, bem como, esta tecnologia vem agregando cada vez mais serviços a um sistema de telefonia, porém com uma pequena limitação se comparada a VoiP com a telefonia convencional, nas questão de estrutura física, mas fornecendo segurança na transmissão das chamadas.

	Na  terceira etapa, relatara as funcionalidades da ferramenta eleita, bem como, alguns recursos avançados desta, e como essa ferramenta agregara valor ao ambiente de telefonia para o seu emprego.

	Na quarta etapa, mostrara, os resultados, ou seja, as falhas técnicas, ou mesmo humanas apresentados pelos atendentes através do emprego do questionário, bem como, pelo período de permanência para mapear o fluxo operacional e realizar à analise de requisitos para o emprego e configuração da ferramenta com eficiência e qualidade no teleatendimento.

	Na quinta etapa, sucederá a instalação do sistema operacional, das bibliotecas e dependências do Asterisk, bem como, sua própria instalação e configuração para à aplicação do protótipo do servidor VoIP, e ainda, coletar os resultados ao aplicarmos a ferramenta no teleatendimento.


\newpage
%======================== Cronograma ==============================
\section{Cronograma}
\thispagestyle{empty}

\begin{center}
\begin{table}[h]
	\footnotesize
    \begin{tabular}{|c|c|c|c|p{0.05cm}|p{0.05cm}|p{0.05cm}|p{0.05cm}|p{0.05cm}|p{0.05cm}|p{0.05cm}|p{0.05cm}|p{0.05cm}|p{0.05cm}|p{0.05cm}|p{0.05cm}|p{0.05cm}|p{0.05cm}|p{0.05cm}|p{0.05cm}|p{0.05cm}|p{0.05cm}|p{0.05cm}|p{0.05cm}|}
    \hline
    \multicolumn{4}{|c|}{\textbf{Meses}} & \multicolumn{4}{c|}{\textbf{Fev}} & \multicolumn{4}{c|}{\textbf{Mar}} & \multicolumn{4}{c|}{\textbf{Abri}} & \multicolumn{4}{c|}{\textbf{Maio}} & \multicolumn{4}{c|}{\textbf{Junho}} \\ \hline
    \multicolumn{4}{|c|}{\textbf{Etapas/Semanas}} &\tiny1 &\tiny2 &\tiny3 &\tiny4 &\tiny1 &\tiny2 &\tiny3 &\tiny4 &\tiny1 &\tiny2 &\tiny3 &\tiny4 &\tiny1 &\tiny2 &\tiny3 &\tiny4 &\tiny1 &\tiny2 &\tiny3 &\tiny4 \\ \hline
    \multicolumn{4}{|c|}{\multirow{2}{*}{\textbf{Plano de Trabalho}} } & \multicolumn{20}{c|}{} \\ 
    \multicolumn{4}{|c|}{} & \multicolumn{20}{c|}{} \\ \hline
    \multicolumn{4}{|c|}{\multirow{2}{*}{Elaboração do Plano de Trabalho} } &\colorbox{red}{} &\colorbox{red}{} &\colorbox{red}{} &\colorbox{red}{} & & & & & & & & & & & & & & & & \\ \cline{5-24}
    \multicolumn{4}{|c|}{} &\colorbox{blue}{} &\colorbox{blue}{} &\colorbox{blue}{} &\colorbox{blue}{} & & & & & & & & & & & & & & & & \\ \hline
    \multicolumn{4}{|c|}{\multirow{2}{*}{Entrega do Plano de Trabalho} } & & & & &\colorbox{red}{} &\colorbox{red}{} & & & & & & & & & & & & & & \\ \cline{5-24}
    \multicolumn{4}{|c|}{} & & & & &\colorbox{blue}{} &\colorbox{blue}{} & & & & & & & & & & & & & & \\ \hline
    
    \multicolumn{4}{|c|}{\multirow{2}{*}{\textbf{Preâmbulo}} } & \multicolumn{20}{c|}{} \\ 
    \multicolumn{4}{|c|}{} & \multicolumn{20}{c|}{} \\ \hline
    \multicolumn{4}{|c|}{\multirow{2}{*}{1. Introdução} } & & & & & & &\colorbox{red}{} &\colorbox{red}{} & & & & & & & & & & & & \\ \cline{5-24}
    \multicolumn{4}{|c|}{} & & & & & & &\colorbox{blue}{} &\colorbox{blue}{} & & & & & & & & & & & & \\ \hline
    \multicolumn{4}{|c|}{\multirow{2}{*}{1.1 Objetivos} } & & & & & & & & &\colorbox{red}{} &\colorbox{red}{} & & & & & & & & & & \\ \cline{5-24}
    \multicolumn{4}{|c|}{} & & & & & & & & &\colorbox{blue}{} &\colorbox{blue}{} & & & & & & & & & & \\ \hline
    \multicolumn{4}{|c|}{\multirow{2}{*}{1.2 Justificativa e Motivação} } & & & & & & & & &\colorbox{red}{} &\colorbox{red}{} &\colorbox{red}{} & & & & & & & & & \\ \cline{5-24}
    \multicolumn{4}{|c|}{} & & & & & & & & &\colorbox{blue}{} &\colorbox{blue}{} &\colorbox{blue}{} & & & & & & & & & \\ \hline
        \multicolumn{4}{|c|}{\multirow{2}{*}{1.3 Metodologia} } & & & & & & & & &\colorbox{red}{} &\colorbox{red}{} &\colorbox{red}{} & & & & & & & & & \\ \cline{5-24}
    \multicolumn{4}{|c|}{} & & & & & & & & &\colorbox{blue}{} &\colorbox{blue}{} &\colorbox{blue}{} & & & & & & & & & \\ \hline
    \multicolumn{4}{|c|}{\multirow{2}{*}{1.5 Organização do Texto} } & & & & & & & & & &\colorbox{red}{} &\colorbox{red}{} &\colorbox{red}{} & & & & & & & & \\ \cline{5-24}
    \multicolumn{4}{|c|}{} & & & & & & & & & &\colorbox{blue}{} &\colorbox{blue}{} &\colorbox{blue}{} & & & & & & & & \\ \hline
    
    \multicolumn{4}{|c|}{\multirow{2}{*}{\textbf{2. Revisão Bibliográfica}} } & \multicolumn{20}{c|}{} \\ 
    \multicolumn{4}{|c|}{} & \multicolumn{20}{c|}{} \\ \hline
    \multicolumn{4}{|c|}{\multirow{2}{*}{2.1 Telefonia convencional} } & & & & & & & & & & & & & & & &\colorbox{red}{} &\colorbox{red}{} & & & \\ \cline{5-24}
    \multicolumn{4}{|c|}{} & & & & & & & & & & & & & & & &\colorbox{blue}{} &\colorbox{blue}{} & & & \\ \hline
    \multicolumn{4}{|c|}{\multirow{2}{*}{2.2 Voz sobre IP (VoIP)} } & & & & & & & & & & & & & & & & & &\colorbox{red}{} &\colorbox{red}{} &\colorbox{red}{} \\ \cline{5-24}
    \multicolumn{4}{|c|}{} & & & & & & & & & & & & & & & & & & & & \\ \hline
    
    \multicolumn{4}{|c|}{\multirow{2}{*}{\textbf{Resultados Preliminares}} } & & & & & & & & & & & & &\colorbox{red}{} &\colorbox{red}{} &\colorbox{red}{} &\colorbox{red}{} & & & & \\ \cline{5-24}
    \multicolumn{4}{|c|}{} & & & & & & & & & & & & &\colorbox{blue}{} &\colorbox{blue}{} &\colorbox{blue}{} &\colorbox{blue}{} & & & & \\ \hline
    
    \multicolumn{4}{|c|}{\multirow{2}{*}{\textbf{Implementações}} } & & & & & & & & & & & & &\colorbox{red}{} &\colorbox{red}{} &\colorbox{red}{} &\colorbox{red}{} & & & & \\ \cline{5-24}
    \multicolumn{4}{|c|}{} & & & & & & & & & & & & &\colorbox{blue}{} &\colorbox{blue}{} &\colorbox{blue}{} &\colorbox{blue}{} & & & & \\ \hline
    
    \multicolumn{4}{|c|}{\multirow{2}{*}{\textbf{Entrega da 1 Etapa Concluida}} } & & & & & & & & & & & & & & & & &\colorbox{red}{} &\colorbox{red}{} & & \\ \cline{5-24}
    \multicolumn{4}{|c|}{} & & & & & & & & & & & & & & & & &\colorbox{blue}{} &\colorbox{blue}{} & & \\ \hline
    \end{tabular}
\end{table}

\begin{tabular}{|c c| c c|}
\hline
  Previsto = &\colorbox{red}{} & Realizado = &\colorbox{blue}{} \\ \hline
\end{tabular}

\end{center}

%para nova pagina
\newpage
%==================== Organiza\c{c}\~{a}o do Texto ========================
\section{Organização do Texto}
Este trabalho está estruturado em cinco capítulos, descrito da seguinte forma:

O capítulo 1, descreve de maneira sucinta a abordagem do assunto que será trabalhado, ou seja, o problema apresentado e a escolha da solução a qual sera empregada para a solução deste.

O capítulo 2 apresentara a revisão bibliográfica, ou seja, o referencial teórico, para que se possa elaborar o projeto. 

O capítulo 3 tera a descrição do projeto, deste o sistema operacional eleito para que se possa instalar a ferramenta Asterisk, a sua própria instalação e de suas bibliotecas e dependências, por fim sua configuração para que se possa chegar ao resultado esperado que é o servidor VoIP, para que possa ser empregado no teleatendimento.

O capítulo 4, será dedicado aos testes do servidor VoIP, bem como, a coleta dos resultados para levantamento e analise dos detalhes da ferramenta ao ser aplicada no ambiente do teleatendimento.

O capítulo 5 apresentará as conclusões, assim como algumas sugestões de trabalhos futuros, para a continuidade desta pesquisa.