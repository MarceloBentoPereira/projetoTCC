\setlength{\absparsep}{18pt} % ajusta o espaçamento dos parágrafos do resumo
\begin{resumo}

Os órgãos de segurança pública prestam os mais variados servidos aos cidadãos, porém quando à necessidade de um serviço, nem sempre os órgãos de segurança pública estão presentes fisicamente, e o meio mais eficiente de contactá-los é pelos telefones emergências que estes disponibilizam, logo este estes telefones devem estar imunes a falhas. Portanto este trabalho, tem o propósito de apresentar uma melhoria no teleatendimento emergencial da Polícia Militar de Dourados, para tanto, estará sendo apresentado conceitos de telefonia convencional, VoIP (Voz sobre IP) e de uma ferramenta híbrida, pois, esta terá que implementar as funções de uma central telefônica tradicional e também, protocolos de VoIP, e conter funcionalidades de um PABX para solucionar falhas técnicas deste teleatendimento, assim, proporcionará segurança para os atendentes e confiabilidade para os cidadãos que necessitem destes serviços emergenciais.

 \textbf{Palavras-chaves}: \textit{VoIP. PSTN. Teleatendimento. Agregação de Serviços a Telefonia.}
\end{resumo}
